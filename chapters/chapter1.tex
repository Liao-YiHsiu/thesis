\section{研究動機}
我們身處於一個資訊爆炸的時代,在這個時代當中,資訊增長量是相當驚人的,而這使得傳統一篇一篇吸收資料的方式不再實用,每個人擁有的吸收時間都有限,因此如果沒有足夠好的檢索系統幫助使用者從資料海中尋找到使用者想要的資料,即使這個世界每天產生很多資料意義也不大,因為沒有人有時間去吸收。

語音是人類最自然的溝通方法之一,人類每天透過語音溝通的資料量遠遠大於每天產生的文字量。語音帶有的資訊甚至比文字帶有的資訊還要豐富,人可以從文字得知撰寫者的想法,但如果撰寫者是用語音來記錄這些資訊的話,我們甚至可以透過聲調、語氣得知撰寫者的心情或是態度。近年來隨著科技與網際網路的興起,語音文件量正蓬勃地增加當中,隨著線上影片、會議錄音、線上課程等網站的興起,語音資料量越來越多,因此如果在其中找到使用者感興趣的資料便成為重要的議題,即為語音資訊檢索 (Speech Information
Retrieval)~\cite{chelba2008retrieval, lee2005spoken}。相較於文字資訊檢索,語音資訊檢索面臨到更多的困難,如辨識錯誤、辨識訓練資料不足等問題,使得此問題更形困難。


\section{研究方向}
資訊檢索的目的通常是為了找出有出現查詢詞(Query Terms)的文件。但使用者往往不會滿足於只找到包含查詢詞的文件。通常使用者會希望能找到所有與查詢詞「語意上相關」的文件。比如查詢詞為「東京旅遊」的話,使用者通常會期待查詢到的文件包含「東京住宿」、「東京景點」等結果,此即為語意檢索 (Semantic Retrieval) ,為此篇論文之主軸。

語音檢索傳統的作法為將語音文件辨識為文字檔後,再對文字檔做檢索。但辨識的步驟中需要大量的訓練資料來訓練聲學模型(Acoustic Model)和語音模型(Language Model),而訓練資料尚需有人工的標注,因此獲得這些訓練資料通常是相當困難且昂貴的。另一方面,語音文件每天產生的量很多,只是缺乏足夠的人為標注因此無法利用,所以近年來興起另一個研究領域-非監督式(Unsupervised)語音檢索。非監督式語音檢索過去的作法包括動態時間規劃(Dynamic Time Warping)
可以直接在聲音訊號上比對查詢詞是否有出現,但此篇論文提出一種新的方式,利用自動尋找之語音片段加強監督式語意檢索,並且利用語音片段實作非監督式語意檢索。

% \section{研究貢獻}
% 本論文之主要貢獻如下:
% \begin{itemize}
% \itemsep -2pt
% \end{itemize}
    

\section{章節安排}
本論文章節安排如下:
\begin{itemize}
\itemsep -2pt %reduce space between items
    \item  第二章:介紹本論文相關背景知識。
    \item  第三章:介紹如何以語音片段改善監督式語意檢索。
    \item  第四章:介紹如何以語音片段實現非監督式語意檢索。
    \item  第五章:介紹如何以遞迴式類神經網路語言模型產生之詞向量改善第四章的非監督式語意檢索。
    \item  第六章:介紹如何將本論文之語音檢索系統實作到 Google Glass 上。
    \item  第七章:本論文之結論與未來研究方向。
\end{itemize}

