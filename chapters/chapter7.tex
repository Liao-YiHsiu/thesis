\section{本論文主要的研究貢獻與未來展望}
\subsection{使用聲學片段加強語音文件檢索}
本論文中主要貢獻在於利用聲學片段解決傳統語音檢索的難題,由於傳統的語音文件檢索是先辨識後在詞圖上檢索,但有許多聲學上的資訊在辨識之後就消失了,因此本論文試圖在檢索時加入聲學片段的資訊以提升檢索系統成效,主要貢獻條列如下:

1. 使用聲學片段加強監督式語音文件的語意檢索系統,因為傳統的語音文件檢索是先辨識後在詞圖上檢索,但如果此時有辭典外詞彙或是辨識錯誤的話,檢索結果就會很差了,因此本論文使用聲學片段解決此難題。

2. 使用聲學片段達成非監督式語音文件的語意檢索系統,傳統的語意檢索系統需要先將語音文件辨識成詞圖後才進行語意檢索,但是這樣需要已訓練得很好的聲學模型和語言模型,而這兩者的訓練通常是非常昂貴的,因此我們使用聲學片段解決這個問題。

3. 使用聲學片段雖然能加強以上兩種狀況,但是聲學片段有一個缺點:聲學片段在訓練時並不知道聲音和字之間的關聯,因此會將所有音同字不同的聲音都歸類到同一個聲學片段中,但如此一來在檢索時便無法區別不同的字義了,進而導致檢索的成效很差,所以我們試圖用遞迴式類神經網路語言模型產生的詞表示法來區分出同一個聲學片段中對應到不同字義的聲學片段。

\subsubsection{未來的改進方向}
由於使用聲學片段進行完全非監督式的語音文件檢索是一個十分新穎的研究題目,因此目前系統的平均準確率距離監督式語音文件檢索尚有一段距離,需要更進一步地研究,而可能的研究方向如下:

1. 將同音的聲學片段盡可能按照它對應到的字區分開來:由於目前的聲學片段是將同音的片段盡量分在一起,但這在檢索時會造成困難,因此如果能將聲學片段按照它對應到的字區分開來的話,將能使檢索的效能進步許多,第~\ref{sec:chap5}章 中嘗試了一些做法,但還可以嘗試不同的分群演算法和更好的詞表示法。

2. 使用詞表示法改進檢索系統:目前的檢索系統仍然是使用傳統的詞表示法,因此無法得知詞與詞之間的相似關係,如果能將檢索系統的詞表示法改為 第~\ref{sec:chap5}章中的方法,應能在語意檢索上再進步許多。

\subsection{實作雲端語音辨識與應用程式於 Google 眼鏡}
行動裝置在這幾年間快速地崛起,連帶使得行動裝置上最自然的輸入方式-語音輸入受到許多人的重視,另一方面,行動裝置上的語音辨識系統會遭遇到許多與傳統語音辨識系統不同的挑戰與機會,諸如語速的變化、聲學/語言模型/辭典的個人化、行動裝置上的感測器等等。本論文將本實驗室開發的雲端個人化語音辨識系統實作於目前最流行的穿戴式裝置-Google 眼鏡之上,並於其上建立了兩個應用程序:

1. 雲端個人化語言學習系統:使用雲端語音辨識加上Google翻譯,使學生能夠隨時隨地學習自己需要的內容。

2. 雲端個人化新聞查詢系統:使用雲端語音辨識加上~\ref{sec:chap4_semantic_retrieval}中提到的監督式語意檢索系統,讓使用者隨時隨地都能查詢自己感興趣的新聞。
