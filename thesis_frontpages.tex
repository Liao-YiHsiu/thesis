\NTUtitlepage  % 產生論文封面

\newpage
\setcounter{page}{1}
\pagenumbering{roman}

\NTUoralpage  % 產生口試委員會審定書

\mydoublespacing
\begin{acknowledgement} %誌謝
兩年來的碩士生活,隨著這份論文的誕生而告了一個段落。回思過去兩年來的生活,有做出成果的喜悅、也有處處碰壁時的苦悶、也有跟同學共同奮鬥的記憶,自己在這兩年中實在成長了許多,而這些都要感謝實驗室的大家長:李琳山教授。教授在實驗室營造了自由研究的氛圍,讓實驗室的同學都能按自己喜好自由發展研究方向,並從旁關心協助同學的研究。我在這樣的氛圍下也受益許多,學習到了許多做研究與做人處事的方法。

這些研究能夠順利完成要由衷地感謝我的家人,包含我的父母和妹妹,他們無論何時都支持我的決定,並且從旁給我協助,在我最忙碌而都很晚回家的那段時間,他們也是很包容我,並給予我支持。還有我的女朋友,她也在我最辛苦的時候支持和幫忙我,並常常跟我討論未來的規劃,讓我的決定都做的更好。

接著要感謝的是從大四下就帶領我研究的李宏毅學長。學長在我對語音什麼都不懂的狀況下教我基礎知識,並教導我如何尋找研究方向與撰寫 Paper,一直到最後順利地發表這些論文,都必須大力感謝學長的幫忙。

與實驗室的同學相處的這段時光將是碩士生活中最難忘的一段日子,博士班的學長小安、阿邦、瑪雅、Aaron、青峰哥、宏毅哥在我們還是菜鳥時教導我們很多知識;上一屆的學長蘇培豪、溫宗憲、周宥宇、林博智、陳泰元常常跟我們在實驗室討論跟一起聊天,也恭喜你們最近都有很好的發展;同屆的向思蓉、周伯威、余典翰、鍾承道、蘇嘉雄,我們無論是修課或是研究上都是彼此的好戰友,我從你們身上都學到很多;下一屆的楊子毅、劉元銘、吳全勳、曾柏翔、熊信寬、蔡政昱、魏承寬,你們將是實驗室下一代的主力,祝你們未來研究順利!

最後要感謝我的朋友、同學、以及伙伴們,不論是平常一起吃飯聊天、有正事時的一起奮鬥、或一起出去玩,你們都是我平常生活上最大的支持!

\end{acknowledgement}

\begin{zhAbstract}  %中文摘要
本論文之主軸在探討語音數位內容之語意檢索 (Semantic Retrieval of Spoken Content)。由於近年來網路日新月異,使得網路上包含語音資訊的多媒體數位內容 (Multimedia Content) 如線上課程、電影、戲劇、會議錄音等日漸增加,因此,語音數位內容之檢索也隨之受到重視。但以前的語音數位內容檢索多半著重於口述語彙偵測 (Spoken Term Detection),而本篇論文將把目標放在語意檢索(指找到語意相關的語音文件,但未必包含查詢詞 (Query Terms)),實現的方法主要是借助查詢詞擴展 (Query Expansion),並另外加入了一套自動習得之聲學組型 (Automatically Discovered Acoustic Patterns) 用以解決以往語音數位內容語意檢索之困難。

首先,由於傳統的語音數位內容語意檢索是先將語音文件辨識為以文字構成的詞圖後,再於詞圖上進行查詢詞擴展,但有許多聲學上的資訊會在辨識之中流失,或是有辨識錯誤與辭典外辭彙也會使檢索系統的成效下降,因此本論文在文字的查詢詞擴展之外,再加入一套自動習得之聲學組型的查詢詞擴展,並結合兩套查詢詞擴展之結果回傳給使用者。

此外,使用聲學組型也可以直接達成非監督式 (Unsupervised) 語音文件的語意檢索。傳統的語意檢索必須依賴文字才知語意,故需將語音文件辨識成詞圖,但是這樣需要已訓練得很好的聲學模型和語言模型,而這兩者的訓練需要有妥為標注 (annotated) 並和數位內容適度匹配 (matched) 的訓練語料。通常是非常昂貴的,因此我們將所有語音文件辨識為聲學組型的序列之後,在這些聲學組型的序列上進行查詢詞擴展,進而達到無需標注語料的非監督式語音數位內容之語意檢索。

另一方面,由於聲學組型在訓練時並不知道聲音和詞之間的關聯,所以會將所有同音詞的聲音歸類到同一個聲學組型中,這會使得檢索的成效下降。所以本論文進一步使用遞迴式類神經網路語言模型 (Recurrent Neural Network Language Model) 的詞表示法 (Word Representation) 將同一個聲學組型按照句法 (Syntactics) 和語意 (Semantics) 的不同進一步分群為不同的聲學組型,以便提升檢索系統成效。

最後,由於行動裝置日益重要,也使得行動裝置上的語音輸入漸受重視,因此本論文在 Google 眼鏡上開發了兩個應用程序:雲端個人化語言翻譯系統和雲端個人化新聞查詢系統,幫助使用者在行動裝置上快速地取得想要的資訊。

\end{zhAbstract}

{
%\zhKaiFont
\mysinglespacing\selectfont
\tableofcontents %目錄

\listoffigures  %圖目錄

\listoftables  %表目錄
\par
}

\newpage
\setcounter{page}{1}
\pagenumbering{arabic}
