\NTUtitlepage  % 產生論文封面

\newpage
\setcounter{page}{1}
\pagenumbering{roman}

\NTUoralpage  % 產生口試委員會審定書

\mydoublespacing
%\begin{acknowledgement} %誌謝
   %請在這裡寫您的誌謝辭 
%\end{acknowledgement}

\begin{zhAbstract}  %中文摘要
本論文之主軸在探討如何利用自動習得之聲學片段 (Automatically Discovered Acoustic Patterns) 解決傳統語音檢索的難題。由於傳統的語音文件檢索是先將語音文件辨識後在詞圖上檢索,但有許多聲學上的資訊會在辨識之中流失,或是有辨識錯誤與辭典外辭彙也會使檢索系統成效下降,因此本論文首先試圖在檢索時加入聲學片段的資訊使系統能利用聲學片段的資訊提升檢索系統成效。此外,語意檢索 (Semantic Retrieval)
也是一個重要目標,本論文用查詢詞擴展 (Query Expansion) 來實現它。

此外,使用聲學片段也可以直接達成非監督式 (Unsupervised) 語音文件的語意檢索(指找到語意相關的語音文件,但未必包含查詢詞 (Query Terms))。傳統的語意檢索必須依賴文字才知語意,故需將語音文件辨識成詞圖,但是這樣需要已訓練得很好的聲學模型和語言模型,而這兩者的訓練通常是非常昂貴的,因此我們嘗試用聲學片段及查詢詞擴展直接進行語意檢索。另一方面,由於聲學片段在訓練時並不知道聲音和字之間的關聯,所以會將所有音同字不同的聲音歸類到同一個聲學片段中,這會使得檢索的成效下降。所以本論文進一步使用遞迴式類神經網路語言模型的詞表示法來區分同一個聲學片段是否對應到同樣的字,以便提升檢索系統成效。

最後,由於行動裝置日益重要,也使得行動裝置上的語音輸入漸受重視,因此本論文在 Google 眼鏡上開發了兩個應用程序:雲端個人化語言學習系統和雲端個人化新聞查詢系統,幫助使用者在行動裝置上快速地取得想要的資訊。

\end{zhAbstract}

{
%\zhKaiFont
\mysinglespacing\selectfont
\tableofcontents %目錄

\listoffigures  %圖目錄

\listoftables  %表目錄
\par
}

\newpage
\setcounter{page}{1}
\pagenumbering{arabic}
